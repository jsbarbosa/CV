\documentclass[11pt, letterpaper, sans]{moderncv}        % possible options include font size ('10pt', '11pt' and '12pt'), paper size ('a4paper', 'letterpaper', 'a5paper', 'legalpaper', 'executivepaper' and 'landscape') and font family ('sans' and 'roman')
\usepackage{fontawesome}
% modern themes
\moderncvstyle{banking}                            % style options are 'casual' (default), 'classic', 'oldstyle' and 'banking'
\moderncvcolor{blue}                                % color options 'blue' (default), 'orange', 'green', 'red', 'purple', 'grey' and 'black'
%\renewcommand{\familydefault}{\sfdefault}         % to set the default font; use '\sfdefault' for the default sans serif font, '\rmdefault' for the default roman one, or any tex font name
%\nopagenumbers{}                                  % uncomment to suppress automatic page numbering for CVs longer than one page

% character encoding
\usepackage[utf8]{inputenc}                       % if you are not using xelatex ou lualatex, replace by the encoding you are using
%\usepackage{CJKutf8}                              % if you need to use CJK to typeset your resume in Chinese, Japanese or Korean

% adjust the page margins
\usepackage[scale=0.8]{geometry}
%\setlength{\hintscolumnwidth}{3cm}                % if you want to change the width of the column with the dates
%\setlength{\makecvtitlenamewidth}{10cm}           % for the 'classic' style, if you want to force the width allocated to your name and avoid line breaks. be careful though, the length is normally calculated to avoid any overlap with your personal info; use this at your own typographical risks...

\usepackage{import}
\usepackage{footmisc}

% personal data
\name{Juan Sebasti\'an}{Barbosa}
%\title{Curriculum Vitae}                               % optional, remove / comment the line if not wanted
%\address{Calle 126 \# 52A 92}{Bogot\'a}{Colombia}% optional, remove / comment the line if not wanted; the "postcode city" and and "country" arguments can be omitted or provided empty
\phone[mobile]{316 794 6510}                   % optional, remove / comment the line if not wanted
%\phone[fixed]{6135083}                    % optional, remove / comment the line if not wanted
%\phone[fax]{+3~(456)~789~012}                      % optional, remove / comment the line if not wanted
\email{jsbarbosacoy@hotmail.com}                               % optional, remove / comment the line if not wanted
\homepage{www.github.com/jsbarbosa}                         % optional, remove / comment the line if not wanted
%\extrainfo{additional information}                 % optional, remove / comment the line if not wanted
%\photo[64pt][0.4pt]{picture}                       % optional, remove / comment the line if not wanted; '64pt' is the height the picture must be resized to, 0.4pt is the thickness of the frame around it (put it to 0pt for no frame) and 'picture' is the name of the picture file
%\quote{Some quote}                                 % optional, remove / comment the line if not wanted

% to show numerical labels in the bibliography (default is to show no labels); only useful if you make citations in your resume
%\makeatletter
%\renewcommand*{\bibliographyitemlabel}{\@biblabel{\arabic{enumiv}}}
%\makeatother
%\renewcommand*{\bibliographyitemlabel}{[\arabic{enumiv}]}% CONSIDER REPLACING THE ABOVE BY THIS

% bibliography with mutiple entries
%\usepackage{multibib}
%\newcites{book,misc}{{Books},{Others}}
%----------------------------------------------------------------------------------
%            content
%----------------------------------------------------------------------------------
\begin{document}
%\begin{CJK*}{UTF8}{gbsn}                          % to typeset your resume in Chinese using CJK
%-----       resume       ---------------------------------------------------------
\maketitle

\small{
Candidate to the bachelor degree of Physics and Chemistry with several years of experience as a teaching assistant of courses in Physics and Computational Methods. Skills in identifying and solving problems in multiple areas of knowledge. Motivated by acquiring new knowledge and developing or improving tools that generate positive impact on all types of users. Passionate about the development of knowledge in the world of programming and its application in different areas of research and industry. Ability to work under pressure, within multidisciplinary teams, with leadership skills in the implementation and development of ideas and the subsequent visualization of results.
}
%\small{}

\section{Education}

\vspace{5pt}

\subsection{Academic training}

\vspace{5pt}

\begin{itemize}

\item{\cventry{2013--Present}{Physics student}{Universidad de los Andes}{Colombia}{4.01/5.00}{}}
\item{\cventry{2014--Present}{Chemistry student}{Universidad de los Andes}{Colombia}{4.01/5.00}{}}

\end{itemize}

\vspace{2pt}

\section{Work experience}
	\subsection{Academic}
		\begin{itemize}
			\item
			{
				\cventry{January 2018--Present}{Research assistant}{Universidad de los Andes, Centro de Microscopía}{(Microscopy center)}{}
				{
					\url{https://investigaciones.uniandes.edu.co/es/microscopia/}
					\vspace{3pt}
					\begin{itemize}
						\item Administrative tasks: generation of quotes and user service
						\item Laboratory assistant: management of the Confocal Laser Scanning Microscope, and fluorescence microscopes
						\item Analysis of microscopy images
						\item Developer of the software for the centralization of user requests \href{https://github.com/jsbarbosa/microbill}{(\texttt{Microbill}, \faGithub/microbill, {\color{blue} Python})}
						\item Developer of the software for the data extraction of the Atomic Force Microscope
						\href{https://github.com/jsbarbosa/SPM}{(\texttt{Igor Extractor}, \faGithub/SPM, {\color{blue} Python})}
%						\item Actualmente desarrollando una aplicaci\'on web para la asignaci\'on de reservas con {\color{blue} Django} \href{https://github.com/jsbarbosa/microserver}{(\texttt{microserver}, \faGithub/microserver, {\color{blue} Python})}
					\end{itemize}
				}
			}
			\item
			{
				\cventry{August 2016--December 2017}{Teaching assistant of  Métodos Computacionales}{Universidad de los Andes, Departamento de Física}{(Physics Department, Computational methods)}{}
				{
					\vspace{3pt}
					\begin{itemize}
						\item Proposal of exercises and workshops
						\item Development of the libraries: \href{https://github.com/jsbarbosa/rastrohut}{\texttt{astrohut} (\faGithub/astrohut, {\color{blue} Python})},  \href{https://github.com/jsbarbosa/rippleTank}{\texttt{rippleTank} (\faGithub/rippleTank, {\color{blue} Python})} and a series of computer demonstration experiments \url{https://github.com/ComputoCienciasUniandes/Demonstrations}
					\end{itemize} 
				}
			}
			\vspace{6pt}
			\item
			{
				\cventry{August 2015--May 2016}{Teaching assistant of Física I}{Universidad de los Andes, Departamento de Física}{(Physics Department, Physics 1)}{}
				{
					\vspace{3pt}
					\begin{itemize}
						\item Development of methodologies that allow tracking the failures or progress of students
					\end{itemize}
				}
			}
			\vspace{6pt}
			
			\item{
				\cventry{January 2015--May 2015}{Teaching assistant of Astrobiología}{Universidad de los Andes, Departamento de Física}{(Physics Department, Astrobiology)}{}
				{
					\vspace{3pt}
					\begin{itemize}
						\item Enriching and fundamental experience for the formation and the discovery of potential and interest in the field of teaching and the search for knowledge
					\end{itemize}
				}
			}
		\end{itemize}
	\vspace{6pt}
	\subsection{Industrial}
		\begin{itemize}
			\item
			{
				\cventry{January 2017--Present}{Software Developer}{Tausand Electronics}{}{}
				{
					Tausand is a scientific instrumentation development company. \url{https://www.tausand.com/}
					\vspace{3pt}
					\begin{itemize}
						\item 
						Currently responsible for developing the software \href{https://www.tausand.com/downloads/}{\texttt{Abacus}} ({\color{blue} Python}), which allows to use the coincidence counter that bears the same name
						\item Microcontroller programmer for various applications of the company
%						\item Vinculaci\'on laboral por horas: 18 horas mensuales en promedio
					\end{itemize}
				}
			}
			\vspace{6pt}
		\end{itemize}

\section{Actividades extracurriculares}

	\begin{itemize}
		\item{
			\cventry{Present}{Colaborador, 1/822 coautores}{Sympy}{}{}
			{
				\url{https://github.com/sympy/sympy/blob/master/AUTHORS}
				\vspace{3pt}
				\begin{itemize}
					\item Colaborador ocasional en el desarrollo de la librería \href{https://www.sympy.org/en/index.html}{\color{blue}SymPy} para el cálculo simbólico en {\color{blue}Python}
				\end{itemize}
			}
		}
		\item{
			\cventry{Mayo 27-31 2019}{Participante}{Machine Learning for Quantum Matter and Technology}{Décima Escuela de F\'isica Matem\'atica}{}
			{
				\url{https://matematicas.uniandes.edu.co/~cursillo_gr/escuela2019/index.php}
			}
		}
	
		\item{
			\cventry{Octubre 29-31 2018}{Expositor}{Expociencia, Uniandes en Corferias}{}{}
			{
				\url{https://ingenieria.uniandes.edu.co/Paginas/Noticias.aspx?nid=429}\\
				\url{https://medicina.uniandes.edu.co/index.php/es/facultad/noticias/448-expociencia-2018}
				\vspace{3pt}
				\begin{itemize}
					\item Divulgador de ciencia para el p\'ublico asistente
					\item Presentador de experimentos demostrativos con microscopios del Centro de Microscop\'ia de la Universidad de los Andes
				\end{itemize}
			}
		}
	
		\item{
			\cventry{Abril 29-30 2017}{Organizador}{Python Bootcamp Uniandes, Universidad de los Andes}{}{}
			{
				\url{https://pythonbootcampuniandes.github.io}
				\vspace{3pt}
				\begin{itemize}
					\item Desarrollo y presentación de la soluciones a los ejercicios propuestos durante los talleres
				\end{itemize}
			}
		}
	
		\item{
			\cventry{Julio 2016-Present}{Generador de contenido y entusiasta opensource}{56 proyectos en GitHub}{}{}
			{
				\url{https://github.com/jsbarbosa}
				\vspace{3pt}
				\begin{itemize}
					\item Creyente de la comunidad opensource. En parte soy, gracias a aquellos que divulgaron sus proyectos, y me permitieron aprender de ellos. Razón por la cual publico mis proyectos de tiempo libre, con la esperanza de retribuir un poco la generosidad de la comunidad que tanto me ha dado
					\item \textbf{Usuario frecuente de:}
					\href{https://jupyter.org/}{Jupyter}, \href{https://www.numpy.org/}{NumPy}, \href{https://www.matplotlib.com}{Matplotlib}, \href{https://www.scipy.org/}{SciPy}, \href{https://www.sympy.org/en/index.html}{SymPy}, \href{https://pandas.pydata.org/}{Pandas} y \href{https://www.qt.io/}{PyQt}. \textbf{Con frecuencia media y baja:} \href{https://python-pillow.org/}{Pillow}, \href{https://scikit-image.org/}{scikit-image}, \href{https://scikit-learn.org/stable/}{scikit-learn}, \href{https://www.djangoproject.com/}{Django}
				\end{itemize}
			}
		}
	\end{itemize}

\section{Habilidades}
\vspace{6pt}

\begin{itemize}
	\item \textbf{Programación:}
	\begin{itemize}
		\item \textbf{Avanzado:} C, Python, \LaTeX, y Bash
		\item \textbf{Intermedio:} Java
	\end{itemize} 	
	\vspace{6pt}
	\item \textbf{Comunicación:} Inglés flu\'ido. Certificado de IELTS (7.0/9.0).
\end{itemize}

\vfill
\enlargethispage{\footskip}
\footnotetext{\faGithub{} hace referencia a \url{https://github.com/jsbarbosa}}

\end{document}

